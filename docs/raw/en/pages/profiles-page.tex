% SPDX-License-Identifier: GPL-3.0-or-later OR CC-BY-SA-4.0
\section{Profiles Page}\label{sec:profiles-page} %%##$section-title>>
%%!!intro<<
Profiles page can be accessed from the \hyperref[subsec:main-page-options-menu]{options-menu} in the
main page. It primarily displays a list of configured profiles along with the typical options to
perform operations on them. New profiles can also be added using the \textit{plus} button at the
bottom-right corner. Profiles can be imported, duplicated or deleted. Clicking on a profile item
opens its \hyperref[sec:profile-page]{profile page}.
%%!!>>

\subsection{Options Menu}\label{subsec:profiles-options-menu} %%##$options-menu-title>>
%%!!options-menu<<
The three-dots menu in the top-right corner opens the global option menu. It has an option to import
an existing profile that was previously exported from App Manager.

Long clicking on any profile item brings up another options-menu. It offers the following options:
\begin{itemize}
    \item \textbf{Apply now\dots.} This option can be used to apply the profile directly. When
    clicked, a dialog is displayed where it is possible to select a \hyperref[subsubsec:profile-state]{profile state}.
    On selecting one of the states, the profile will be applied immediately.

    \item \textbf{Delete.} Clicking on this option will remove the profile immediately without a warning.

    \item \textbf{Duplicate.} This option can be used to duplicate the profile. When clicked, a
    dialog is displayed where it is possible to set a name for the new profile. On clicking ``OK'',
    the \hyperref[sec:profile-page]{profile page} will be loaded by duplicating all the
    configurations that this profile have. However, the profile will not be saved until it is saved
    manually.

    \item \textbf{Copy profile ID.} This option is used to copy the unique profile ID of the profile.
    The profile ID can be used to \hyperref[subsec:triggering-a-profile]{trigger the profile} from a
    third-party application.

    \item \textbf{Export.} Export the profile to an external storage. Profiles exported this way can
    be imported via the \textit{import} option as mentioned above.

    \item \textbf{Create shortcut.} This option can be used to create a shortcut for the profile.
    There are two options: \textit{Simple} and \textit{Advanced}. When configured with the latter
    option, it prompts the user to select a profile state when the shortcut is invoked. The former
    option, on the other hand, always uses the default state that was configured when the profile
    was last saved.
\end{itemize}
%%!!>>
